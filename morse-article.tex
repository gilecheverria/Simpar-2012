\documentclass{llncs}
%\documentclass[letterpaper, 10pt]{article}


\usepackage{graphicx}
\usepackage{subfig}
\usepackage{url}
\usepackage{listings}
\usepackage{authblk}


\newcommand{\TODO}[1]{\textbf{TODO:} \emph{#1} }
\newcommand{\NOTE}[1]{\textbf{NOTE:} \emph{#1} }
% Uncomment the following lines to remove TODO and NOTE labels
\renewcommand{\TODO}[1]{}
\renewcommand{\NOTE}[1]{}

\def\GenoM{G\textsuperscript{en}oM~}

\lstset{language=Python, basicstyle=\scriptsize, tabsize=4}



\title{\LARGE \bf Evaluation of Robotics Software Using the MORSE Simulator}

%%%%%%%%%%%%%%%%%%%%%%%%%%%%%%%%%%%%%%%%%%%%%
% Standard author list
%%%%%%%%%%%%%%%%%%%%%%%%%%%%%%%%%%%%%%%%%%%%%
%\author[1]{Gilberto Echeverria \thanks{gechever@laas.fr}}
%\author[1]{S{\'e}verin Lemaignan \thanks{slemaign@laas.fr}}
%\author[1]{Arnaud Degroote \thanks{adegroot@laas.fr}}
%\author[2]{Michael Karg\thanks{kargm@in.tum.de}}
%\author[3]{Pierrick Koch\thanks{pierrick.koch@gmail.com}}
%\author[4]{Charles Lesire-Cabaniols\thanks{charles.lesire@onera.fr}}
%\affil[1]{LAAS-CNRS, Toulouse, France}
%\affil[2]{TUM, Munich, Germany}
%\affil[3]{IRD, Hanoi, Vietnam}
%\affil[4]{ONERA, Toulouse, France}
%
%\renewcommand\Authands{ and }

%%%%%%%%%%%%%%%%%%%%%%%%%%%%%%%%%%%%%%%%%%%%%
% LLNCS author list
%%%%%%%%%%%%%%%%%%%%%%%%%%%%%%%%%%%%%%%%%%%%%
\author{Gilberto Echeverria\inst{1}\thanks{\email{gechever@laas.fr}}
    \and S{\'e}verin Lemaignan\inst{1}\thanks{\email{slemaign@laas.fr}}
    \and Arnaud Degroote\inst{1}\thanks{\email{adegroot@laas.fr}}
    \and Simon Lacroix\inst{1}\thanks{\email{slacroix@laas.fr}}
    \and Michael Karg\inst{2}\thanks{\email{kargm@in.tum.de}}
    \and Pierrick Koch\inst{3}\thanks{\email{pierrick.koch@gmail.com}}
    \and Charles Lesire-Cabaniols\inst{4}\thanks{\email{charles.lesire@onera.fr}}
}

\institute{
	    CNRS, LAAS, 7 avenue du colonel Roche, F-31077 Toulouse, France
	    Universit{\'e} de Toulouse, UPS, INSA, INP, ISAE, LAAS,
	    F-31077 Toulouse, France
        \and
        Technische Universit{\"a}t M{\"u}nchen,
        Arcisstrasse 21, 80333 M{\"u}nchen, Germany
        \and
        IRD,
        Hanoi, Vietnam
        \and
       	ONERA Centre de Toulouse -- DCSD, 2 avenue {\'E}douard Belin,
    	F-31055 Toulouse, France
}


\begin{document}
\maketitle

\begin{abstract}
  MORSE is a robotics simulation software developed with the collaboration of
  researchers in several universities. It is a tool to test robotics software
  and algorithms in fairly complex environments. The simulations allow a medium
  to high level of abstraction, to enable researchers to focus on the solution
  of complex tasks.  This is accomplished by connecting directly with the
  robotics software using various existing middlewares, such as ROS, YARP and
  others.
  MORSE makes use of the Blender 3D software to produce realistic looking
  environments with physics simulation.  After three years of development,
  MORSE is a mature tool with a large collection of components.

  We present in this paper the current state of the simulator as well as the
  use cases where it has been used to validate several robotics algorithms.
\end{abstract}

%%%%%%%%%%%%%%%%%%%%%%%%%%%%%%%%%%%%%%%%%%%%%%%%%%%%%%%%%%%%%%%%%%%%%%
\section{Introduction}
\label{section:introduction}

Citing of Gazebo: \cite{Koenig04designand}.



\section{Blender Implementation}
\label{section:blender}

MORSE is developed as a library of Python scripts that run on the Blender Game Engine (BGE).
The BGE offers a powerful environment for a graphical application with a variety of events.
Blender uses the Bullet Physics Engine to simulate collisions between objects, which requires configuring the properties of each element, such as mass, friction coefficient, collision bounding box and other parameters.

MORSE is organised as a main core of control functionality that sets up and coordinates the events in the BGE, and a collection of components that can be used to assemble a robot. Additionally, there is a whole set of communication tools that allow each of the MORSE components to connect with external robotics software via middlewares.

Individual components are minimalistic in their functionality and  completely middleware agnostic. They store the important data that needs to be shared outside MORSE in a Python dictionary called \texttt{local\_data}.
When a simulation scene is created, the components are linked to middlewares as specified in the configuration script, and the necessary functionality is added to the components to be able to transmit/receive data through the middleware.
During a simulation, each component performs its expected task and exchanges the \texttt{local\_data} with the external software.



The animation armatures of Blender can also be used inside the BGE, along with the IK solver iTasc \cite{iTaSC}. This is used for robotics arms and for a virtual human avatar.
The human avatar allows a user to interact with the robots in the simulated environment. It can be controlled much like a character in a videogame, using either the mouse and keyboard or a combination of the Microsoft Kinect and the Nintendo Wiimote. The avatar can navigate around the environment, sit down on chairs and take objects specifically marked as ``graspable''.

\section{Simulated Components}
\label{section:components}

\begin{itemize}
  \item Robots
  \item Sensors
  \item Actuators
\end{itemize}


\section{Use Cases}

\begin{itemize}
  \item Outdoors simulation (Rosace and Action)
  \item Indoors simulation (KUL)
  \item Human interaction (Kinect + Wiimote)
  \item Flying robots
\end{itemize}


\section{New Features}

\begin{itemize}
  \item Builder API
  \item Multi-node
  \item Simulated or abstract wheels
  \item Middleware configurations
\end{itemize}

%%%%%%%%%%%%%%%%%%%%%%%%%%%%%%%%%%%%%%%%%%%%%%%%%%%%%%%%%%%%%%%%%%%%%%
\section{Summary}
\label{section:discussion}


MORSE is developed as an open--source project, the source code can be
downloaded from the GIT repository:
(\url{http://github.com/laas/morse.git})

User documentation and additional information is also available at
(\url{http://morse.openrobots.org})


\subsubsection*{Acknowledgments}
This work has been partially supported by the DGA founded Action project
(\url{http://action.onera.fr}) and the STAE foundation Rosace project\\
(\url{http://www.fondation-stae.net})

% ---- Bibliography ----
\bibliographystyle{unsrt}
\bibliography{morseBiblio}
\end{document}
